a causa de boa parte dos nossos sofrimentos é a nossa ignorância em relação à natureza espiritual de Tudo: todos os objetos, nossos próprios corpos, nossas relações com as outras pessoas e com nossa forma de experienciar o tempo.

pensamos haver separação onde não há. separamos a vida espiritual da vida ``normal''. buscamos práticas meditativas para tratar sintomas como ansiedade e depressão, quando na verdade o problema é bem mais profundo. ``quero meditar para voltar à minha vida normal com menos estresse''. claro, todo motivo para praticar meditação é válido. mas até mesmo o próprio motivo é algo a ser transcendido, pois a meditação paleativa não é suficiente para nos libertar dos ciclos de sofrimento.

durante a meditação, simplesmente reproduzimos os mesmos vícios mentais do resto do dia. nos deixamos levar pela mente que supervaloriza objetivos e que insiste em julgar a nós mesmos e aos outros incessantemente.

é ingenuidade achar que aqueles minutinhos meditando, por si, vão curar seja lá o que for. ``ah, minha vida espiritual é quando vou ao centro de meditação''; ``minha vida espiritual é quando vou praticar yoga''; ``minha vida espiritual é quando vou à igreja''. todos exemplos de manifestação do mesmo erro.

à medida que vamos deixando dissolver a barreira que criamos entre a vida espiritual e a vida normal, o estado meditativo vai se tornando cada vez mais presente. sem os vícios e os condicionamentos mentais incessantes, eventualmente chegamos ao ponto no qual podemos simplesmente nos permitir sentar ao chão sem objetivos nem julgamentos. a prática se torna um deleite. a vida se torna um deleite. isto é iluminação.
