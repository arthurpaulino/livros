\chapter[(Ashtavakra)]{Desidentificação (Ashtavakra)}

[9.1] Forças opostas --- coisas feitas e por fazer --- quando elas terminam? E para quem? Levando isto em consideração, desconstrua seus desejos. Deixe que as coisas se vão. Seja indiferente ao mundano.

[9.2] Raro e abençoado é aquele cujo apego aos prazeres e ao conhecimento mundano foram extintos ao serem observados os caminhos egoistas da humanidade.

[9.3] Ao perceber o mundano como fonte de sofrimento triplo, o sábio man\-tém-se estático. Ao mundo, insubstancial, transitório e impermanente, só resta a observação

[9.4] Houve algum período no qual a humanidade pôde existir sem os opostos? Esqueça os opostos. Seja contente com o que vier. Perfeição.

[9.5] Como é possível não alcançar a verdadeira indiferença e a paz ao perceber a diferença de opiniões entre os grandes videntes, santos e yogis?

[9.6] Não seria um verdadeiro professor aquele que, por meio da desidentificação com o mundano, da serenidade e da razão vê sua verdadeira natureza?

[9.7] Nas inúmeras formas do universo, perceba a vacuidade essencial. Você encontrará a libertação em direção ao Eu instantaneamente.

[9.8] O desejo cria a ilusão. Renuncie. Renuncie os desejos e o mundano ilusório. Assim você poderá vivenciar quem você realmente é.
