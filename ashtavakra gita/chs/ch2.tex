\chapter[(Janaka)]{Alegria na Autorrealização (Janaka)}

[2.1] Sou realmente puro e em paz. Consciência além da causalidade natural. Por muito tempo me deixei aflingir pela delusão.

[2.2] Assim como eu dou luz a esse corpo, também o faço com o Universo. Dessa forma, todo o Universo é meu ou, de forma alternativa, nada é.

[2.3] Agora que me desidentifiquei do corpo e das coisas, pela grande Graça, a Verdade se revela.

[2.4] Assim como ondas, espuma e bolhas não se diferenciam da água, tudo que emana do Eu não se diferencia do Eu.

[2.5] Observe atentamente o tecido e você verá apenas fios. Observe atentamente a criação e você verá apenas o Eu.

[2.6] Assim como o açúcar produzido da cana de açúcar é permeado com a doçura, tudo isto, expressão do Eu, é completamente permeado por mim.

[2.7] Da ignorância sobre si a ilusão surge. No autoconhecimento a ilusão cessa. Uma corda não é uma serpente, embora possa parecer.

[2.8] Luz é a minha verdadeira natureza, nem menos nem mais que isto. O Universo se manifesta ao meu vislumbre.

[2.9] A ilusão da ignorância surge em mim assim como a cobra surge na corda, a miragem de água no horizonte ensolarado e a prata na madrepérola.
